\documentclass{book}

\usepackage[T1]{fontenc}
%\usepackage{lmodern}
%\usepackage{mathpazo}
\usepackage{courier}
%\usepackage{cmbright}
%\usepackage{ccfonts}
%\usepackage{fouriernc}

\usepackage{fancyhdr}
\pagestyle{fancy}
\usepackage{titlesec}

\renewcommand{\partmark}[1]{ \markboth{#1}{} }

\fancyhf{}
\fancyhead[CE]{ \nouppercase{\leftmark} }
\fancyhead[CO]{ \nouppercase{\rightmark}}

\usepackage{amsmath}
\usepackage{amsthm}

% Use (A),(B),... for equations, not (1),(2),...
\numberwithin{equation}{subsection}
\renewcommand{\theequation}{\Alph{equation}}

\newcommand{\half}{ \frac{1}{2} }

\newcommand{\realpart}{\mathrm{Re}}
\newcommand{\imagpart}{\mathrm{Im}}

\newcommand{\eps}{\epsilon}
\newcommand{\cosec}{\mathrm{cosec}}
%\newcommand{\cosec}{\csc}

\newcommand{\hardY}{\mathsf{Y}} %TODO:improve %17.6

\newcommand{\absval}[1]{\left| #1 \right|}
\newcommand{\thebrace}[1]{\left\{ #1 \right\} }
\newcommand{\thebracket}[1]{\left[ #1 \right] }
\newcommand{\theparen}[1]{\left( #1 \right) }

\newcommand{\wandwsectionsymbol}{\S}
\newcommand{\wandwsubsectionsymbol}{\ensuremath{\cdot}}
\newcommand{\hardchapterref}[1]{\wandwsectionsymbol #1}
\newcommand{\hardsectionref}[2]{\wandwsectionsymbol #1\wandwsubsectionsymbol#2}
\newcommand{\hardsubsectionref}[3]{\wandwsectionsymbol #1\wandwsubsectionsymbol#2#3}
\newcommand{\hardsubsubsectionref}[4]{\wandwsectionsymbol #1\wandwsubsectionsymbol#2#3#4}

% Add W&W-style end-of-line example citations
\newcommand{\addexamplecitation}[1]{#1 TODO}

\newcommand{\dmeasure}{\, d}

\newcommand{\corollary}{Corollary}

\theoremstyle{remark}
\newtheorem{wandwexample}{Example}
\numberwithin{wandwexample}{subsubsection}

%%%%%%%%%%%%%%%%%%%%%%%%%%%%%%%%%%%%%%%%%%%%%%%%%%
%% Names
%%%%%%%%%%%%%%%%%%%%%%%%%%%%%%%%%%%%%%%%%%%%%%%%%%
\newcommand{\Schlafli}{Schl\"afli}
\newcommand{\Schlomilch}{Schl\"omilch}

\setcounter{chapter}{16}

\begin{document}
\part{The Transcendental Functions}
%355
\chapter{Bessel Functions}
 \section{The Bessel coefficients}
 In this chapter we shall consider a class of functions known as  \emph{Bessel functions}
 or \emph{cylindrical functions}
 which have many analogies with the Legendre functions of  Chapter XV.
 Just as the Legendre functions proved to be particular forms of the
 hypergeometric function with three regular singularities,  so the
 Bessel functions are particular forms of the  confluent
 hypergeometric function with one regular and one irregular
 singularity. As is the case of the Legendre functions, we first
 introduce\footnote{This procedure is due to TODO \Schlomilch}
 a certain set of the Bessel functions as coefficients in an
 expansion.

 For all values of $z$ and $t$ ($t=0$ excepted), the function
 $$
 e^{ \frac{1}{2} z \left( t - \frac{1}{t} \right)}
 $$
can be expanded by Laurent's theorem in a series of positive and
negative powers of $t$. If the coefficient of $t^n$, where $n$ is any
integer positive or negative, be denoted by $J_n(z)$, it follows, from 
\hardsectionref{5}{6}, that 
$$
J_n(z) = \frac{1}{2\pi i} \int^{(0+)} u^{-n-1} e^{\half z \theparen{u
    - \frac{1}{u}}} \dmeasure u.
$$

To express $J_n(z)$ as a power series in $z$, write $u = 2t/z$; then
$$
J_n(z) = \frac{1}{2\pi i} \theparen{\half z }^n \int^{(0+)} t^{-n-1}
\exp \thebrace{ t - \frac{z^2}{4t}  } \dmeasure t
$$
since the contour is any one which encircles the origin once 
counter-clockwise, we may take it to be the circle 
$\absval{t}=1$; as the integrand can be expanded in a series of powers
of $z$ uniformly convergent on this contour, it follows from
\hardsectionref{4}{7} that 
$$
J_n(z)
=
\frac{1}{2\pi i}
\sum_{r=0}^{\infty} \theparen{\half z}^{n+2r} 
\int^{(0+)} t^{-n-r-1} e^t \dmeasure t.
$$

Now the residue of the integrand at $t=0$ is 
$\thebrace{ (n+r)!  }^{-1}$ by 
\hardsectionref{6}{1}, when $n+r$ is a positive 
integer or zero; when $n+r$ is a negative integer 
the residue is zero.

Therefore, if $n$ is a positive integer or zero, 
\begin{align*}
J_n(z) =& \sum_{r=0}^{\infty} \frac{ (-)^r (\half z)^{n+2r} }{
  r!(n+r)! } \\
=& \frac{ z^n }{ 2^n n! } 
\thebrace{ 1 - \frac{z^2}{2^2.1(n+1)} + \frac{z^4}{2^4.1.2(n+1)(n+2)}
  - \cdots };
\end{align*}
%
%356
%
whereas, when $n$ is a negative integer equal to $-m$,
$$
J_n(z)
= \sum_{r=m}^{\infty} \frac{ (-)^r (\half z)^{2r-m} }{ r!(r-m)!  }
  = \sum_{s=0}^{\infty} \frac{ (-)^{m+s} (\half z)^{m+2s}  }{ (m+s)! s!  },
  $$
  and so $J_n(z) = (-)^m J_m(z)$.

The function $J_n(z)$, which has now been defined for all integral
values of $n$, positive and negative, is called the
\emph{Bessel coefficient} of order $n$; the series defining it 
converges for all values of $z$.

%\begin{smalltext}
We shall see later (\hardsectionref{17}{2}) that Bessel
coefficients are a particular case of a class of functions known as
\emph{Bessel functions}.

The series by which $J_n(z)$ is defined occurs in a memoir by Euler,
on the vibrations of a stretch circular membrane, TODO, 
an investigation dealt with below in \hardsubsectionref{18}{5}{1};
is also occurs in a memoir by Lagrange on elliptic motion, TODO.

The earliest systematic study of the functions was made in 1824 by
Bessel in his TODO; special cases of Bessel coefficients had, however,
appeared in researches published before 1769; the earliest of these is
in a latter, dated Oct. 3, 1703, from Jakob Bernolli to
Leibniz\footnote{TODO}, in which occurs a series which is now
described as a Bessel function of order $\half$; the Bessel
coefficient of order zero occurs in 1732 in Daniel Bernoulli's memoir
on the oscillations of heavy chains, TODO.

In reading some of the earlier papers on the subject, it should be
remembered that the notation has changed, what was formerly called
$J_n(z)$ being now written $J_n(2z)$.
% \end{smalltext}
\begin{wandwexample}
Prove that if 
$$
\frac{ 2b(1+\theta^2)  }{ (1-2a\theta-\theta^2)^2 + 4b^2\theta^2  }
=
A_1 + A_2 \theta + A_3 \theta^2 + \cdots,
$$
then
$$
e^{az} \sin bz 
=
A_1 J_1(z) + A_2 J_2(z) + A_3 J_3(z) + \cdots.
$$
\addexamplecitation{Math. Trip. 1896.}
\end{wandwexample}

[For, if the contour $D$ in the $u$-plane be a circle with centre
$u=0$ and radius large enough to include the zeros of the denominator,
we have
$$
e^{\half z \theparen{u - \frac{1}{u}}} 
\frac{ 2b \theparen{\frac{1}{u^2} + \frac{1}{u^4} }  }{ \theparen{1 -
    \frac{2a}{u} - \frac{1}{u^2}}^2
  + \frac{4 b^2}{u^2}  }
=
\sum_{n=1}^{\infty}
e^{\half z \theparen{u - \frac{1}{u}}} A_n u^{-n-1},
$$
the series on the right converging uniformly on the controu; and so, 
using \hardsectionref{4}{7} and replacing the integrals by Bessel
coefficients, we have
\begin{align*}
  \frac{1}{2\pi i}
  \int_D
  e^{\half z \theparen{u - \frac{1}{u}}}
  \frac{ 2b \theparen{\frac{1}{u^2} + \frac{1}{u^4} }  }{ \theparen{1 -
      \frac{2a}{u} - \frac{1}{u^2}}^2
    + \frac{4 b^2}{u^2}  }
  \dmeasure u
  =& \frac{1}{2\pi i}
  \int_D
  e^{\half z \theparen{u - \frac{1}{u}}}
  \theparen{ \frac{A_1}{u^2} + \frac{A_2}{u^3} + \frac{A_3}{u^4} +
    \cdots  }
  \dmeasure u \\
  =& A_1 J_1(z) + A_2 J_2(z) + A_3 J_3(z) + \cdots .
\end{align*}

%
%357
%
In the integral on the left write $\half (u - u^{-1}) - a = t$, so that as
$u$ describes a circle of radius $e^{\beta}$, $t$ describes an ellipse
with 
semiaxes $\cosh\beta$ and $\sinh\beta$ with foci at 
$-a \pm i$; then we have
$$
\sum_{n=1}^{\infty}
=
\frac{1}{2\pi i}
\int
\frac{ e^{z(t+a)}b \dmeasure t  }{ t^2 + b^2  },
$$
the contour being the ellipse just specified, which contains the zeros
of $t^2 + b^2$. Evaluating the integral by 
\hardsectionref{6}{1}, we have the required result.]
\begin{wandwexample}
  Shew that, when $n$ is an integer,
  $$
  J_n(y+z)
  =
  \sum_{m = -\infty}^{\infty} J_{m}(y) J_{n-m}(z)
  $$
\addexamplecitation{K. Neumann and \Schlafli}
\end{wandwexample}
[Consider the expansion of each side of the equation
$$
\exp \thebrace{ \half (y+z) \theparen{t - \frac{1}{t}}  }
=
\exp \thebrace{ \half y \theparen{t - \frac{1}{t}}  }
\cdot
\exp \thebrace{ \half z \theparen{t - \frac{1}{t}}  }.]
$$
\begin{wandwexample}
  Shew that 
$$
e^{i z \cos \phi}
= 
J_0(z)
+ 2 i \cos\phi J_1(z)
+ 2 i^2 \cos 2\phi J_2(z)
+ \cdots.
$$
\end{wandwexample}
\begin{wandwexample}
  Shew that if $r^2 = x^2 + y^2$
  $$
  J_0(r)
  =
  J_0(x)J_0(y)
  - 2 J_2(x) J_2(y)
  + 2 J_4(x) J_4(y)
  - \cdots .
  $$
\addexamplecitation{K. Neumann and Lommel.}
\end{wandwexample}

\subsection{Bessel's differential equation}
We have seen that, when $n$ is an integer, the Bessel coefficient of
order $n$ is given by the formula
$$
J_n(z)
=
\frac{1}{2\pi i} 
\theparen{\half z}^n
\int^{(0+)}
t^{-n-1}
\exp\theparen{ t - \frac{z^2}{4t}  }
\dmeasure t.
$$

From this formula we shall now shew that $J_n(z)$ is a solution of the
linear differential equation
$$
\frac{d^2 y}{dz^2}
+ \frac{1}{z} \frac{dy}{dz}
+ \theparen{ 1 - \frac{n^2}{z^2}  } y
= 0,
$$
which is called Bessel's equation for functions of order $n$.

For we find on performing the differentiations 
(\hardsectionref{4}{2}) that
\begin{align*}
\frac{d^2 J_n(z)}{dz^2}
+& \frac{1}{z} \frac{d J_n(z)}{dz}
+ \theparen{ 1 - \frac{n^2}{z^2}  } J_n(z)
 \\
%
=& 
\frac{1}{2\pi i} 
\theparen{ \half z  }^n 
\int^{(0+)}
t^{-n-1}
\thebrace{ 1 - \frac{n+1}{t} + \frac{z^2}{4t^2}  }
\exp\theparen{ t - \frac{z^2}{4t}  }
\dmeasure t \\
%
=& 
-\frac{1}{2\pi i}
\theparen{ \half z  }
\int^{(0+)}
\frac{d}{dt} \thebrace{ t^{-n-1} \exp\theparen{ t - \frac{z^2}{4t}  }
}
\dmeasure t \\
%
=& 0,
\end{align*}
since $t^{-n-1} \exp\theparen{ t - z^2/4t  }$ is one-valued.
\emph{Thus we have proved that }
$$
\frac{d^2 J_n(z)}{dz^2}
+ \frac{1}{z} \frac{d J_n(z)}{dz}
+ \theparen{ 1 - \frac{n^2}{z^2}  } J_n(z) 
= 0.
$$
The reader will observe that $z=0$ is a regular point and 
$z = \infty$ an irregular point, all other points being ordinary
points of this equation.
%
%358
%
\begin{wandwexample}
  By differentiating the expansion
  $$
  e^{\half z (t - \frac{1}{t})} = \sum_{n=-\infty}^{\infty} t^n J_n(z)
  $$
  with regard to $z$ and with regard to $t$, shew that the Bessel
  coefficients satisfy Bessel's equation.
\addexamplecitation{St John's, 1899.}
\end{wandwexample}
\begin{wandwexample}
  The function $P_n^m\theparen{ 1 - \frac{z^2}{2n^2}  }$ satisfies the
  equation defined by the scheme
  $$
  TODO
  $$
  shew that $J_m(z)$ satisfies the confluent form of this equation
  obtained by making $n \rightarrow \infty$.
\end{wandwexample}
\section{The solution of Bessel's equation when $n$ is not necessarily
  an integer.}
We now proceed, after the manner of \hardsectionref{15}{2}, to
extend the definition of $J_n(z)$ to the case when $n$ is any number,
real or complex. It appears by methods similar to those of
\hardsubsectionref{17}{1}{1} that, for all values of $n$, the
equation
$$
\frac{d^2 y}{dz^2}
+ \frac{1}{z} \frac{dy}{dz}
+ \theparen{ 1 - \frac{n^2}{z^2}  } y
= 0
$$
is satisfied by an integral of the form
$$
y
=
z^n
\int_C
t^{-n-1}
\exp \theparen{ t - \frac{z^2}{4t}  }
\dmeasure t
$$
provided that $t^{-n-1}\exp(t - z^2/4t)$ resumes its initial value
after describing $C$ and that differentiations under the sign of
integration is justified.

Accordingly, we define $J_n(z)$ by the equation
$$
J_n(z) 
= 
\frac{z^n}{2^{n+1}\pi i}
\int_{-\infty}^{(0+)}
t^{-n-1} 
\exp\theparen{ t - \frac{z^2}{4t}  }
\dmeasure t,
$$
the expression being rendered precise by giving
$\arg z$ its principal value and taking
$\absval{ \arg t } \leq \pi$ on the contour.

To express this integral as a power series, we observe that it is an
analytic function of $z$; and we may obtain the coefficients in the
Taylor's series in powers of $z$ by differentiating under the sign of 
integration
(\hardsectionref{5}{32} and \hardsubsectionref{4}{4}{4}
TODO:print has two section symbols together).
Hence we deduce that
\begin{align*}
  J_n(z)
  =& \frac{z^n}{2^{n+1}\pi i}
  \sum_{r=0}^{\infty} \frac{ (-)^r z^{2r}  }{ 2^{2r} r!  }
  \int_{-\infty}^{(0+)}
  e^t
  t^{-n-r-1}
  \dmeasure t
  \\
  %
  =&
  \sum_{r=0}^{\infty}
  \frac{ (-)^r z^{n+2r}  }{ 2^{n+2r} r! \Gamma(n+r+1)  },
\end{align*}
by \hardsubsectionref{12}{2}{2}. This is the expansion in question.

%
%359
%
\emph{Accordingly, for general values of $n$, we define the
  \emph{Bessel function} $J_n(z)$ by the equations}
\begin{align*}
  J_n(z)
  =& \frac{1}{2\pi i} \theparen{\half z}^n
  \int_{-\infty}^{(0+)}
  t^{-n-1}
  \exp\theparen{ t - \frac{z^2}{4t}  }
  \dmeasure t
  \\
  %
  =&
  \sum_{r=0}^{\infty}
  \frac{ (-)^r z^{n+2r}  }{ 2^{n+2r} r! \Gamma(n+r+1)  }.
\end{align*}

This function reduces to a Bessel coefficient when $n$ is an integer;
it is sometimes called a Bessel function \emph{of the first kind}.

The reader will observe that since Bessel's equation is unaltered by
writing $-n$ for $n$, fundamental solutions are $J_n(z)$, $J_{-n}(z)$,
except when $n$ is an integer, in which case the solutions are not
independent. With this exception, \emph{the general solution of
  Bessel's equation is
$$
\alpha J_n(z) + \beta J_{-n}(z),
$$
where $\alpha$ and $\beta$ are arbitrary constants.}

A second solution of Bessel's equation when $n$ is an integer will be
given later (\hardsectionref{17}{6}).

\subsection{The recurrence formulae for the Bessel functions.}
As the Bessel function satisfies a confluent form of the
hypergeometric equation, it is to be expected that recurrence formulae
will exist, corresponding to the relations between contiguous
hypergeometric functions indicated in \hardsectionref{14}{7}.

To establish these relations for general values of $n$, real or
complex, we have recourse to the result of \hardsectionref{17}{2}.
On writing the equation
$$
0 = \int_{-\infty}^{(0+)} \frac{d}{dt} \thebrace{ t^{-n}
  \exp\theparen{ t - \frac{z^2}{4t}  }  } \dmeasure t
$$
at length, we have
\begin{align*}
0 =& \int_{-\infty}^{(0+)}
\theparen{ t^{-n} + \frac{1}{4}z^2 t^{-n-2} - nt^{-n-1}  }
  \exp\theparen{ t - \frac{z^2}{4t}  }  \dmeasure t \\
%
  =& 2\pi i
  \thebrace{ (2z^{-1})^{n-1} J_{n-1}(z)
    + \frac{1}{4} (2z^{-1})^{n+1} J_{n+1}(z)
  - n (2z^{-1})^n J_n(z)},
\end{align*}
and so
\begin{equation}
J_{n-1}(z) + J_{n+1}(z) = \frac{2n}{z} J_n(z) 
\label{eq:bessel:recur}
%TODO: fill with dots?
\end{equation}

Next we have, by \hardsubsectionref{4}{4}{4},
\begin{align*}
  \frac{d}{dz} \thebrace{ z^{-n} J_n(z)  }
  =& \frac{1}{2^{n+1}\pi i}
  \frac{d}{dz}
  \int_{-\infty}^{(0+)}
  t^{-n-1}
  \exp\theparen{ t - \frac{z^2}{4t} }
  \dmeasure t
  \\
  %
  =& - \frac{z}{2^{n+2}\pi i}
  \int_{-\infty}^{(0+)}
  t^{-n-2}
  \exp\theparen{ t - \frac{z^2}{4t} }
  \dmeasure t
  \\
  %
  =& -z^{-n} J_{n+1}(z),  
\end{align*}
%
%360
%
and consequently, if primes denote differentiations with regard to
$z$,
\begin{equation}
  J'_n(z) = \frac{n}{z} J_n(z) - J_{n+1}(z)
\label{eq:bessel:deriv}
\end{equation}

From \eqref{eq:bessel:recur} and \eqref{eq:bessel:deriv} it is easy to
derive the other recurrence formulae
\begin{equation}
  J'_{n}(z) = \half \thebrace{ J_{n-1}(z) - J_{n+1}(z)  },
\end{equation}
and
\begin{equation}
  J'_{n}(z) = J_{n-1}(z) - \frac{n}{z} J_n{z}.
\end{equation}
\begin{wandwexample}
  Obtain these results from the power series for $J_n(z)$.
\end{wandwexample}
\begin{wandwexample}
  Shew that 
  $$
  \frac{d}{dz} \thebrace{ z^n J_n(z) } = z^n J_{n-1}(z).
  $$
\end{wandwexample}
\begin{wandwexample}
  Shew that
  $$
  J'_0(z) = -J_1(z).
  $$
\end{wandwexample}
\begin{wandwexample}
  Shew that
  $$
  TODO = J_{n-4}(z) - 4 J_{n-2}(z) + 6 J_n(z) - 4 J_{n+2}(z) + J_{n+4}(z).
  $$
\end{wandwexample}
\begin{wandwexample}
  Shew that
  $$
  J_2(z) - J_0(z) = 2 J''_0(z).
  $$
\end{wandwexample}
\begin{wandwexample}
  Shew that
  $$
  J_2(z) = J''_0(z) - z^{-1} J'_0(z).
  $$
\end{wandwexample}

\subsubsection{Relation between two Bessel functions whose orders
  differ by an integer.}
From the last article can be deduced an equation connecting any two
Bessel functions whose orders differ by an integer, namely
$$
z^{-n-r} J_{n+r}(z) TODO: verify subscript
=
(-)^r
\frac{d^r}{ (z dz)^r }
\thebrace{ z^{-n} J_n(z)  },
$$
where $n$ is unrestricted and $r$ is any positive integer. This result
follows at once by induction from formula \eqref{eq:bessel:deriv},
when it is written in the form
$$
z^{-n-1} J_{n+1}(z) 
=
- \frac{d}{z dz}
\thebrace{ z^{-n} J_n(z)  }.
$$
\subsubsection{The connexion between $J_n(z)$ and $W_{k,m}$
  functions.}
The reader will verify without difficulty that, if in Bessel's
equation we write $y = z^{-\half} v$ and then write $z = x/2i$, we get
$$
\frac{d^2 v}{dx^2}
+
\theparen{ -\frac{1}{4} + \frac{\frac{1}{4} - n^2}{x^2}  } v
=
0,
$$
which is the equation satisfied by $W_{0,n}(x)$; it follows that
$$
J_n(z) = A z^{-\half} M_{0,n}(2iz) + B z^{-\half} M_{0,-n}(2iz).
$$
Comparing the coefficients of $z^{\pm n}$ on each side we see that
$$
J_n(z) = \frac{z^{-\half}}{2^{2n+\half} i^{n+\half} \Gamma(n+1)} M_{0,n}(2iz),
$$
%
%361
%
except in the critical cases when $2n$ is a negative integer; 
when $n$ is half of a negative integer, the result follows from
Kummer's second formula (\hardsubsectionref{16}{1}{1}).
\subsection{The zeros of Bessel functions whose order $n$ is real.}
The relations of \hardsubsectionref{17}{2}{1} enable us to deduce the
interesting theorem that \emph{between any two consecutive real zeros
  of $z^{-n}J_n(z)$, there lies one and only one zero\footnote{TODO}
  of $z^{-n}J_{n+1}(z)$.}

For, from relation TODO:addref when written in the form
$$
z^{-n} J_{n+1}(z) 
= 
- \frac{d}{dz} \thebrace{ z^{-n} J_n(z) },
$$
it follows from Rolle's theorem\footnote{TODO} that between each
consecutive pair of zeros of $z^{-n}J_n(z)$ there is at least one zero
of $z^{-n} J_{n+1}(z)$.

Similarly, from relation TODO:addref when written in the form
$$
z^{n+1} J_n(z) 
= 
\frac{d}{dz} \thebrace{ z^{n+1} J_{n+1}(z) },
$$
it follows that between each consecutive pair of zeros of 
$z^{n+1}J_{n+1}(z)$ there is at least one zero of 
$z^{n+1}J_n(z)$.

Further $z^{-n}J_n(z)$ and 
$\frac{d}{dz} \thebrace{ z^{-n} J_n(z)  }$ have no common zeros; for
the former function satisfies the equation
$$
z \frac{d^2 y}{dz^2} + (2n+1) \frac{dy}{dz} + zy = 0,
$$
and it is easily verified by induction on differentiating this
equation that if both $y$ and $\frac{dy}{dz}$ vanish for any value of
$z$, all differential coefficients of $y$ vanish, and $y$ is zero by 
\hardsectionref{5}{4}.

The theorem required is now obvious except for the numerically
smallest zeros $\pm \xi$ of $z^{-n}J_n(z)$, since (except for $z=0$), 
$z^{-n}J_n(z)$ and $z^{n+1}J_n(z)$ have the same zeros. But $z=0$ is a
zero of $z^{-n}J_{n+1}(z)$, and if there were any other positive zero
of $z^{-n}J_{n+1}(z)$, say $\xi_1$, which was less than $\xi$, then
$z^{n+1}J_n(z)$ would have a zero between $0$ and $\xi_1$, which
contradicts the hypothesis that there were no zeros of 
$z^{n+1}J_n(z)$ between $0$ and $\xi$.

The theorem is therefore proved.

%\begin{smalltext}
[See also \hardsectionref{17}{3} examples TODO and TODO, and example
TODO at the end of the chapter.]
%\end{smalltext}
%
%362
%
\subsection{Bessel's integral for the Bessel coefficients.}
We shall next obtain an integral first given by Bessel in the
particular case of the Bessel functions for which $n$ is a positive
integer; in some respects the result resembles Laplace's integrals
given in \hardsubsectionref{15}{2}{3} and \hardsubsectionref{15}{3}{3}
for the Legendre functions.

In the integral of \hardsectionref{17}{1}, viz, 
$$
J_n(z) 
=
\frac{1}{2\pi i}
\int^{(0+)}
u^{-n-1}
e^{\half z \theparen{ u - \frac{1}{u} }}
\dmeasure u,
$$
take the contour to be the circle $\absval{u} = 1$ and write 
$u = e^{i\theta}$, so that
$$
J_n(z)
= 
\frac{1}{2\pi}
\int_{-\pi}^{\pi}
e^{-ni\theta + iz\sin\theta}
\dmeasure \theta
$$

Bisect the range of integration and in the former part write 
$-\theta$ for $\theta$; we get
$$
J_n(z) 
=
\frac{1}{2\pi}
\int_0^{\pi}
e^{n i \theta - i z \sin\theta}
\dmeasure \theta
+
\frac{1}{2\pi}
\int_0^{\pi}
e^{-n i \theta + i z \sin\theta}
\dmeasure \theta,
$$
and so
$$
J_n(z)
=
\frac{1}{\pi}
\int_0^{\pi}
\cos (n \theta - z \sin \theta)
\dmeasure \theta,
$$
which is the formula in question.
\begin{wandwexample}
  Shew that, when $z$ is real and $n$ is an integer,
  $$
  \absval{ J_n(z) } \leq 1.
  $$
\end{wandwexample}
\begin{wandwexample}
  Shew that, for all values of $n$ (real or complex), the integral
  $$
  y
  =
  \frac{1}{\pi}
  \int_0^{\pi}
  \cos (n \theta - z \sin \theta)
  \dmeasure \theta
  $$
  satisfies
  $$
  \frac{d^2 y}{dz^2}
  +
  \frac{1}{z}
  \frac{dy}{dz}
  +
  \theparen{ 1 - \frac{n^2}{z^2} } y
  =
  \frac{ \sin n\pi }{\pi}
  \theparen{ \frac{1}{z} - \frac{n}{z^2}  },
  $$
  which reduces to Bessel's equation when $n$ is an integer.

[It is easy to shew, by differentiating under the integral sign, that
the expression on the left is equal to
$$
-
\frac{1}{\pi}
\int_0^{\pi}
\frac{d}{d\theta}
\thebrace{ \theparen{ \frac{n}{z^2} + \frac{\cos\theta}{z}  }
\sin(n\theta - z \sin\theta)}
\dmeasure \theta
.]
$$
\end{wandwexample}
%
\subsubsection{The modification of Bessel's integral when $n$ is not
  an integer.}
We shall now shew that\footnote{TODO}, for general values of $n$,
\begin{equation}
J_n(z) 
=
\frac{1}{\pi}
\int_0^{\pi} \cos(n\theta - z\sin\theta) \dmeasure\theta
-
\frac{\sin n\pi}{\pi}
\int_0^{\infty} e^{-n\theta - z\sinh\theta} \dmeasure\theta,
\end{equation}
when $\realpart(z) > 0$. This obviously reduces to the result of
\hardsubsectionref{17}{2}{3} when $n$ is an integer.

Taking the integral of \hardsectionref{17}{2}, viz,
$$
J_n(z) 
=
\frac{z^n}{2^{n+1}\pi i}
\int_{-\infty}^{(0+)}
t^{-n-1}
\exp\theparen{ t - \frac{z^2}{4t}  }
\dmeasure t,
$$
%
%363
%
and supposing that $z$ is positive, we have, on writing
$t = \half u z$,
$$
J_n(z) 
= 
\frac{1}{2\pi i} 
\int_{-\infty}^{(0+)}
u^{-n-1}
\exp \thebrace{ \half z \theparen{ u - \frac{1}{u}  }  }
\dmeasure u.
$$

But, if the contour be taken to be that of the figure consisting of
the real axis from $-1$ to $-\infty$ taken twice and the circle 
$\absval{u} = 1$, this integral represents an analytic function of $z$
when $\realpart(zu)$ is negative as
$\absval{u} \rightarrow \infty$ on the path, \emph{i.e.} when 
$\absval{ \arg z } < \half \pi$; and so, by the theory of analytic
continuation, the formula (which has been proved by a direct
transformation for \emph{positive} values of $z$) is true whenever 
$\realpart(z) > 0$.

Hence
$$
J_n(z)
=
\frac{1}{2\pi i}
\thebrace{
  \int_{-\infty}^{-1}
  +
  \int_C
  +
  \int_{-1}^{-\infty}
}
u^{-n-1}
\exp\thebrace{ \half z \theparen{ u - \frac{1}{u}  }  }
\dmeasure u,
$$
where $C$ denotes the circle $\absval{u} = 1$, and 
$\arg u = -\pi$ on the first path of integration while 
$\arg u = +\pi$ on the third path.

TODO:missingfigure

Writing $u = t e^{\mp \pi i}$ in the first and third integrals
respectively (so that in each case $\arg t = 0$), and 
$u = e^{i\theta}$ in the second, we have
$$
J_n(z) 
=
\frac{1}{2\pi}
\int_{-\pi}^{\pi}
e^{-ni\theta + iz\sin\theta}
\dmeasure \theta
+
\thebrace{
  \frac{e^{(n+1)\pi i}}{2\pi i}
  -
  \frac{e^{-(n+1)\pi i}}{2\pi i}
}
\int_1^{\infty}
t^{-n-1}
e^{\half z \theparen{-t + \frac{1}{t}}}
\dmeasure t.
$$

Modifying the former of these integrals as in
\hardsubsectionref{17}{2}{3} and writing 
$e^{\theta}$ for $t$ in the latter, we have at once
$$
J_n(z)
=
\frac{1}{\pi}
\int_0^{\pi}
\cos(n\theta - z\sin\theta)
\dmeasure\theta
+
\frac{\sin(n+1)z}{\pi}
\int_0^{\infty}
e^{-n\theta - z\sinh\theta}
\dmeasure \theta,
$$
which is the required result, when
$\absval{\arg z} < \half \pi$.

%\begin{smalltext}
When $\absval{ \arg z }$ lies between $\half \pi$ and $\pi$, since 
$J_n(z) = e^{\pm n\pi i} J_n(-z)$, we have
\begin{equation}
  J_n(z)
  =
  \frac{e^{\pm n \pi i}}{\pi}
  \thebrace{
    \int_0^{\pi} \cos(n\theta + z\sin\theta) \dmeasure\theta
    -
    \sin n\pi \int_0^{\infty} e^{-n\theta + z\sinh\theta} \dmeasure\theta
    },
\end{equation}
%\end{smalltext}
the upper or lower sign taken as
$\arg z > \half \pi$ or $< -\half \pi$.

When $n$ is an integer TODO:addref reduces at once to Bessels'
integral, and TODO:addref does so when we make use of the equation 
$J_n(z) = (-)^n J_{-n}(z)$, which is true for integer values of $n$.

%
%364
%
Equation TODO:addref, as already stated, is due to TODO:addcitation,
and equation TODO:addref was given by TODO:addcitation.

These trigonometric integrals for the Bessel functions may be regarded
as corresponding to Laplace's integrals for the Legendre functions.
For (\hardsubsectionref{17}{1}{1} example TODO:addref)
$J_m(z)$ satisfies the confluent form (obtained by making 
$n \rightarrow \infty$ of the equation for
$P_n^m(1-z^2/2n^2)$.

But Laplace's integral for this function is a multiple of 
$$
TODO
$$

The limit of the integrand as $n \rightarrow \infty$ is 
$e^{iz\cos\phi}\cos m\phi$, and this exhibits the similarity of
Laplace's integral for $P_n^m(z)$ to the Bessel-\Schlafli\ integral for 
$J_m(z)$.
\begin{wandwexample}
  From the integral
  $J_0(x) = \int_{-\pi}^{\pi} e^{-ix\cos\phi}\dmeasure\phi$, by a
  change of order of integration, shew that, when $n$ is a positive
  integer and $\cos\theta > 0$,
  $$
  P_n(\cos\theta)
  =
  \frac{1}{\Gamma(n+1)}
  \int_0^{\infty}
  e^{-x\cos\theta}
  J_0(x\sin\theta)
  x^n
  \dmeasure x.
  $$
  TODO:addattribution
\end{wandwexample}
\begin{wandwexample}
  Shew that, with Ferrers' definition of $P_n^m(\cos\theta)$,
  $$
  P_n^m(\cos\theta)
  =
  \frac{1}{\Gamma(n-m+1)}
  \int_0^{\infty}
  e^{-x\cos\theta}
  J_m(x\sin\theta)
  x^n
  \dmeasure x
  $$
  when $n$ and $m$ are positive integers and
  $\cos\theta > 0$.
  TODO:addattribution
\end{wandwexample}
\subsection{Bessel functions whose order is half an odd integer.}
We have seen (\hardsectionref{17}{2}) that when the order $n$ of a
Bessel function $J_n(z)$ is half an odd integer, the difference of the
roots of the indicial equation at $z=0$ is $2n$, which is an integer.
We now shew that, in such cases, $J_n(z)$ is expressible in terms of
elementary functions.

For 
$$
J_{\half}(z) 
= 
\frac{2^{\half} z^{\half}}{\pi^{\half}}
\thebrace{ 1
  - \frac{z^2}{2.3}
  + \frac{z^4}{2.3.4.5}
  - \cdots
}
= \theparen{ \frac{2}{\pi z}  }^{\half}
\sin z,
$$
and therefore (\hardsubsubsectionref{17}{2}{1}{1}) if $k$ is a
positive integer
$$
J_{k+\half}(z)
=
\frac{(-)^k (2z)^{k+\half}}{\pi^{\half}}
\frac{d^k}{d (z^2)^k}
\theparen{ \frac{\sin z}{z}  }
$$
On differentiating out the expression on the right, we obtain the
result that
$$
J_{k+\half}(z) = P_k \sin z + Q_k \cos z,
$$
where $P_k$, $Q_k$ are polynomials in $z^{-\half}$.
\begin{wandwexample}
  Shew that
  $$
  J_{-\half}(z) = \theparen{ \frac{2}{\pi z}}^{\half} \cos z
  $$
\end{wandwexample}
%
%365
%
\begin{wandwexample}
  Prove by induction that if $k$ be an integer and $n = k + \half$,
  then
  \begin{align*}
    J_n(z)
    =
    \theparen{\frac{2}{\pi z}}^{\half}
    \left[
      \cos(z - \half n \pi - \frac{1}{4}\pi)
      \thebrace{
        1
        +
        \sum_{r=1}
        \frac{(-)^r (4n^2-1^2)(4n^2-3^2)\cdots(4n^2-(4r-1)^2)}{(2r)! 2^{6r} z^{2r}}
        }
      \right.
      \\
      % &
      \left.
      +
      \sin(z - \half n \pi - \frac{1}{4}\pi)
      \sum_{r=1}
      \frac{(-)^r (4n^2-1^2)(4n^2-3^2)\cdots(4n^2-(4r-3)^2)}{(2r-1)!
        2^{6r-3} z^{2r-1}}
      \right],
    \end{align*}
    the summations being continued as far as the terms with the
    vanishing factors in the numerators.
\end{wandwexample}
\begin{wandwexample}
  Shew that
  $$
  z^{k+\half} \frac{d^k}{d (z^2)^k} \theparen{\frac{\cos z}{z}}
  $$
  is a solution of Bessel's equation for $J_{k+\half}(z)$.
\end{wandwexample}
\begin{wandwexample}
  Shew that the solution of
  $
  z^{m+\half} \frac{d^{2m+1} y}{dz^{2m+1}} + y = 0
  $
  is
  $$
  y
  =
  z^{\half m + \frac{1}{4}}
  \sum_{p=0}^{2m}
  c_p
  \thebrace{ J_{-m-\half}(2a_pz^{\half}) + iJ_{m+\half}(2a_pz^{\half})},
  $$
  where
  $c_0,c_1,\ldots,c_{2m}$ are arbitrary and
  $a_0,a_1,\ldots,a_{2m}$ are the roots of
  $a^{2m+1}=i$. TODO:addattribution
\end{wandwexample}
\section[Hankel's contour integral for $J_n(z)$.]{Hankel's contour integral\footnote{TODO} for $J_n(z)$.}
Consider the integral
$$
y = z^n \int_A^{(1+,-1-)} (t^2-1)^{n-\half} \cos(zt) \dmeasure t,
$$
where $A$ is a point on the right of the point $t=1$, and
$$
\arg(t-1) = \arg(t+1) = 0
$$
at $A$; the contour may conveniently be regarded as being in the shape
of a figure of eight.

We shall shew that this integral is a constant multiple of $J_n(z)$.
It is easily seen that the integrand returns to its initial value
after $t$ has described the path of integration; for
$(t-1)^{n-\half}$ is multiplied by a factor
$e^{(2n-1)\pi i}$ after the circuit $(1+)$ has been described, and
$(t+1)^{n-\half}$ is multiplied by the factor
$e^{-(2n-1)\pi i}$ after the circuit $(-1-)$ has been described.

Since
$$
\sum_{r=0}^{\infty}
\frac{(-)^r (zt)^{2r}}{(2r)!}
(t^2-1)^{n-\half}
$$
converges uniformly on the contour, we have (\hardsectionref{4}{7})
$$
y
=
sum_{r=0}^{\infty}
\frac{(-)^r z^{n+2r}}{(2r)!}
\int_A^{(1+,-1-)}
t^{2r}
(t^2-1)^{n-\half}
\dmeasure t.
$$

To evaluate these integrals, we observe firstly that they are analytic
functions of $n$ for all values of $n$, and secondly that, when
$\realpart\theparen{n+\half} > 0$, we may deform the contour into the
circles $\absval{t-1}=\delta$, $\absval{t+1}=\delta$ and the real axis
joining the points $t = \pm (1-\delta)$ taken twice, and then we may
make $\delta \rightarrow 0$; the integrals round the circle tend to
zero and, assigning to $t-1$
%
% 366
%
and $t+1$ their appropriate arguments on the modified path of
integration, we get, if $\arg (1-t^2) = 0$ and $t^2 = u$,
\begin{align*}
  \int_A^{(1+,-1-)}
  t^{2r} (t^2-1)^{n-\half} \dmeasure t
  \\
  =&
  e^{(n-\half) \pi i}
  \int_1^{-1} t^{2r} (1-t^2)^{n-\half} \dmeasure t
  + e^{-(n-\half) \pi i}
  \int_{-1}^1 t^{2r} (1-t^2)^{n-\half} \dmeasure t
  \\
  =&
  -4i
  \sin\theparen{n-\half} 
  \pi
  \int_0^1 t^{2r} (1-t^2)^{n-\half} \dmeasure t
  \\
  =&
  -2i \sin\theparen{n-\half}
  \pi
  \int_0^1 u^{r-\half} (1-u)^{n-\half} \dmeasure u
  \\
  =&
  2i \sin\theparen{ n + \half }
  \pi
  \left.
    \Gamma \theparen{r + \half}
    \Gamma \theparen{ n + \half  }
  \right/
  \Gamma \theparen{ n + r + 1  }.
\end{align*}

Since the initial and final expressions are analytic functions of $n$
for all values of $n$, it follows from \hardsectionref{5}{5} that this
equation, proved when
$$
\realpart \theparen{ n + \half } > 0,
$$
is true for all values of $n$.

Accordingly
\begin{align*}
  y
  =&
  \sum_{r=0}^{\infty}
  \frac{(-)^r z^{n+2r} 2i \sin(n+\half) \pi \Gamma(r+\half)
    \Gamma(n+\half)}{ (2r)! \Gamma(n+r+1)}
  \\
  =&
  2^{n+1}
  i
  \sin\theparen{n+\half}
  \pi
  \Gamma\theparen{n+\half}
  \Gamma(\half)
  J_n(z),
\end{align*}
on reduction.

\emph{Accordingly, when
  $
  \thebrace{\Gamma\theparen{\half-n}}^{-1} \neq 0,
  $
  we have
  }
$$
J_n(z)
=
\frac{ \Gamma(\half-n) (\half z)^n }{ 2\pi i \Gamma(\half)}
$$

\corollary. When $\realpart(n+\half) > 0$, we may deform the path of
integration, and obtain the result
\begin{align*}
  J_n(z)
  =&
  \frac{ (\half z)^2 }{ \Gamma(n+\half) \Gamma(\half) }
  \int_{-1}^1 (1-t^2)^{n-\half} \cos(zt) \dmeasure t
  \\
  =&
  \frac{2 . (\half z)^n }{ \Gamma(n+\half) \Gamma(\half) }
  \int_0^{\half \pi} \sin^{2n}\phi \cos(z \cos\phi) \dmeasure\phi.
\end{align*}
\begin{wandwexample}
  Shew that, when $\realpart(n+\half) > 0$,
  $$
  J_n(z)
  =
  \frac{ (\half z)^n }{ \Gamma(n+\half) \Gamma(\half) }
  \int_0^{\pi} e^{\pm iz\cos\phi} \sin^{2n}\phi \dmeasure\phi.
  $$
\end{wandwexample}
\begin{wandwexample}
  Obtain the result
  $$
  J_n(z)
  =
  \frac{ (\half z)^n }{ \Gamma(n+\half) \Gamma(\half) }
  \int_0^{\pi} \cos(z\cos\phi) \sin^{2n}\phi \dmeasure\phi,
  $$
  when $\realpart(n) > 0$, by expanding in powers of $z$ and
  integrating
  (\hardsectionref{4}{7}) term-by-term.
\end{wandwexample}
%
% 367
%
\begin{wandwexample}
  Shew that when $-\half < n < \half$, $J_n(z)$ has an infinite number
  of real zeros. [Let $z = (m+\half)\pi$ where $m$ is zero or a
  positive integer; then by the corollary above
  $$
  J_n(m\pi + \half\pi)
  =
  \frac{z^n}{2^{n-1} \Gamma(n+\half) \Gamma(\half)}
  \thebrace{ \half u_0 - u_1 + u_2 - \cdots + (-)^m u_m},
  $$
  where
  \begin{align*}
    u_r
    =&
    \absval{
      \int_{\frac{2r-1}{2m+1}}^{\frac{2r+1}{2m+1}}
      (1-t^2)^{n-\half} \cos \thebrace{(m+\half)\pi t}
      \dmeasure t
    }
    \\
    =&
    \int_0^{1/(m+\half)}
    \thebrace{
      1
      -
      \theparen{t + \frac{2r-1}{2m+1}}^2
    }^{n-\half}
    \sin\thebrace{ (m+\half)\pi t }
    \dmeasure t,
  \end{align*}
  so, since $n - \half < 0$, $u_m > u_{m-1} > u_{m-2} > \ldots$, and
  hence $J_n(m\pi + \half\pi)$ has the sign of $(-)^m$.
  This method of proof for $n=0$ is due to Bessel.]
\end{wandwexample}
\begin{wandwexample}
  Shew that if $n$ be real, $J_n(z)$ has an infinite number of real
  zeros; and find an upper limit to the numerically smallest of them.
  [Use example TODO combined with \hardsubsectionref{17}{2}{2}.]
\end{wandwexample}

\section{Connexion between Bessel coefficients and Legendre
  functions.}
We shall now establish a result due to Heine\footnote{TODO} which
renders precise the statement of \hardsubsectionref{17}{1}{1} example
TODO, concerning the expression of Bessel coefficients as limiting
forms of hypergeometric functions.

When $\absval{\arg(1\pm z)} < \pi$, $n$ is unrestricted and $m$ is a
positive integer, it follows by differentiating the formula of
\hardsubsectionref{15}{2}{2} that, with Ferrers' definition of
$P_n^{m}(z)$,
$$
P_n^{m}(z)
=
\frac{\Gamma(n+m+1)}{2^m . m! \Gamma(n-m+1)}
(1-z)^{\half m}
(1+z)^{\half m}
F(-n+m,n+1+m; m+1; \half - \half z),
$$
and so, if $\absval{\arg z} < \half\pi$,
$\absval{\arg (1 - \frac{1}{4} z^2/n^2)} < \pi$, we have
$$
P_n^{m}\theparen{1 - \frac{z^2}{2n^2}}
=
\frac{\Gamma(n+m+1) z^m n^{-m}}{2^m . m! \Gamma(n-m+1)}
\theparen{1 - \frac{z^2}{4n^2}}^{\half m}
F(-n+m,n+1+m;m+1; \frac{1}{4} z^2 n^{-2}).
$$

Now make $n \rightarrow +\infty$ ($n$ being positive, but not
necessarily integral), so that, if $\delta = n^{-1}$, 
$\delta \rightarrow 0$ continuously through positive values.

Then
$ \frac{\Gamma(n+m+1)n^{-m}}{\Gamma(n-m+1) n^m} \rightarrow 1$,
by \hardsectionref{13}{6}, and
$ \theparen{1 - \frac{z^2}{4n^2}}^{\half m} \rightarrow 1$.

Further, the $(r+1)$th term of the hypergeometric series is
$$
\frac{
  (-)^r
  (1-m\delta)
  (1+\delta+m\delta+r\delta)
  \thebrace{ 1 - (m+1)^2 \delta^2 }
  \thebrace{ 1 - (m+2)^2 \delta^2 }
  \cdots
  \thebrace{ 1 - (m+r)^2 \delta^2 }
}{ (m+1)(m+2)\cdots(m+r).r!}
(\half z)^{2r};
$$
this is a continuous function of $\delta$ and the series of which this
is the $(r+1)$th term is easily seen to converge uniformly in a range
of values of $\delta$ including the point $\delta=0$; so, by
\hardsubsectionref{3}{3}{2}, we have
\begin{align*}
  \lim_{n \rightarrow \infty}
  \thebracket{ n^{-m} P_n^{m}\theparen{1 - \frac{z^2}{2n^2}}}
  =& \frac{z^m}{2^m.m!}
  \sum_{r=0}^{\infty} \frac{(-)^r (\half
    z)^r}{(m+1)(m+2)\cdots(m+r)r!} \\
  =& J_m(z),
\end{align*}
which is the relation required.
\begin{wandwexample}
  Shew that\footnote{TODO}
  $$
  \lim_{n\rightarrow\infty}
  \thebracket{ n^{-m} P_n^{m}\theparen{\cos \frac{z}{n}}} = J_m(z).
  $$
\end{wandwexample}
%
% 368
%
\begin{wandwexample}
  Shew that Bessel's equation is the confluent form of the equations
  defined by the schemes
  $$
  TODO
  $$
  the confluence being obtained by making $c \rightarrow \infty$.
\end{wandwexample}

\section{Asymptotic series for $J_n(z)$ when $\absval{z}$ is large.}
When have seen (\hardsubsubsectionref{17}{2}{1}{2}) that
$$
J_n(z)
=
\frac{z^{-\half}}{2^{2n+\half} e^{\half(n+\half)\pi TODO} \Gamma(n+1)}
M_{0,n}(2iz),
$$
where it is supposed that 
$\absval{ \arg z } < \pi$,
$-\half\pi < \arg (2iz) < \frac{3}{2} \pi$.

But for this range of values of $z$
$$
M_{0,n}(2iz)
=
\frac{\Gamma(2n+1)}{\Gamma(\half + n)}
e^{(n+\half)\pi i}
W_{0,n}(2iz)
+
\frac{\Gamma(2n+1)}{\Gamma(\half+n)}
W_{0,n}(-2iz)
$$
by \hardsubsectionref{16}{4}{1} example TODO,
if
$-\frac{3}{2} \pi < \arg (-2iz) < \half \pi$; and so, when
$\absval{\arg z} < \pi$,
$$
J_n(z)
=
\frac{1}{(2\pi z)^{\half}}
\thebrace{
  e^{\half (n+\half)\pi i} W_{0,n}(2iz)
  +
  e^{-\half (n+\half)\pi i} W_{0,n}(-2iz).
  }
  $$

But, for the values of $z$ under consideration, the asymptotic
expansion of $W_{0,n}(\pm 2 i z)$ is
\begin{align*}
  e^{\mp i z}
  \left\{
    1
    \pm \frac{(4n^2 - 1^2)}{8iz}
    + \frac{ (4n^2-1^2)(4n^2-3^2)}{2! (8iz)^2}
    \pm \cdots
  \right.
  \\
  \left.
    + \frac{ (\pm 1)^r \thebrace{4n^1-1^2} \thebrace{4n^2-3^2} \cdots
      \thebrace{4n^2 - (2r-1)^2}  }{r! (8iz)^r}
    + O(z^{-r})
    \right\},
  \end{align*}
  and therefore, combining the series, the asymptotic expansion of
  $J_n(z)$, when $\absval{z}$ is large and $\absval{\arg z} < \pi$, is
  \begin{align*}
    TODO
  \end{align*}
  where $U_n(z)$, $-V_n(z)$ have been written in place of the series.

  The reader will observe that if $n$ is half an odd integer these
  series terminate and give the result of \hardsubsectionref{17}{2}{4}
  example TODO.

  %
  % 369
  %
  Even when $z$ is not very large, the value of $J_n(z)$ can be
  computed with great accuracy from this formula. Thus, for all
  positive values of $z$ greater than $8$, the first three terms of
  the asymptotic expansion give the value of $J_0(z)$ and $J_1(z)$ to
  six places of decimals.

  This asymptotic expansion was given by Poisson\footnote{TODO} (for
  $n=0$) and by Jacobi\footnote{TODO} (for general integral values of
  $n$) for real values of $z$.
  Complex values of $z$ were considered by Hankel\footnote{TODO} and
  several subsequent writers. The method of obtaining the expansion
  here given is due to Barnes\footnote{TODO}.

  Asymptotic expansions for $J_n(z)$ when the order $n$ is large have
  been given by Debye TODO and Nicholson TODO.

  An approximate formula for $J_n(nx)$ when $n$ is large and $0<x<1$,
  namely
  $$
  \frac{ x^n \exp\thebrace{n \sqrt{1-x^2}}}{ (2\pi n)^{\half} (1-x^2)^{\frac{1}{4}} \thebrace{1 +
      \sqrt{1 - x^2}}^n},
  $$
  was obtained by Carlini in 1817 in a memoir reprinted in Jacobi's
  TODO.
  The formula was also investigated by Laplace in 1827 in his TODO on
  the hypothesis that $x$ is purely imaginary.

  A more extended account of researches on Bessel functions of large
  order is given by TODO.
  \begin{wandwexample}
    By suitably modifying Hankel's contour integral
    (\hardsectionref{17}{3}), shew that, when
    $\absval{\arg z} < \half \pi$ and
    $\realpart(n+\half) > 0$,
    \begin{align*}
      J_n(z)
      =
      \frac{1}{\Gamma(n+\half) (2\pi z)^{\half}}
      \left[
        e^{i(z - \half n \pi - \frac{1}{4} \pi)}
        \int_0^{\infty}
        e^{-u}
        u^{n-\half}
        \theparen{1 + \frac{iu}{2z}}^{n-\half}
        \dmeasure u
      \right.
      \\
      \left.
        +
        e^{-i(z - \half n \pi - \frac{1}{4}\pi)}
        \int_0^{\infty}
        e^{-u}
        u^{n-\half}
        \theparen{1 - \frac{iu}{2z}}^{n-\half}
        \dmeasure d
        \right];
      \end{align*}
      and deduce the asymptotic expansion of $J_n(z)$ when
      $\absval{z}$ is large and
      $\absval{\arg z} < \half \pi$.

      [Take the contour to be the rectangle whose corners are
      $\pm 1$, $\pm 1 + iN$ the rectangle being indented at $\pm 1$,
      and make $N \rightarrow \infty$; the integrand being
      $(1-t^2)^{n-\half} e^{izt}$.]
    \end{wandwexample}
    \begin{wandwexample}
      Shew that, when $\absval{\arg z} < \half \pi$ and
      $\realpart(n+\half) > 0$,
      $$
      J_n(z)
      =
      \frac{2^{n+1} z^n}{\Gamma(n+\half) \pi^{\half}}
      \int_0^{\half\pi}
      e^{-2z\cot\phi}
      \cos^{n-\half}\phi
      \cosec^{2n+1}\phi
      \sin\thebrace{z - (n-\half)\phi}
      \dmeasure \phi.
      $$
      [Write $u = 2z\cot\phi$ in the preceding example.]l
    \end{wandwexample}
    \begin{wandwexample}
      Shew that, if $\absval{\arg z} < \half \pi$ and
      $\realpart(n+\half) > 0$, then
      $$
      A e^{iz} z^n
      \int_0^{\infty}
      v^{n-\half}
      (1+iv)^{n-\half}
      e^{-2vz}
      \dmeasure v
      +
      B e^{-iz} z^n
      \int_0^{\infty}
      e^{n-\half}
      (1-iv)^{n-\half}
      e^{-2vz}
      \dmeasure v
      $$
      is a solution of Bessel's equation.

      Further, determine $A$ and $B$ so that this may represent
      $J_n(z)$.
      TODO:attribution
    \end{wandwexample}

    \section{The second solution of Bessel's equation when the order
      is an integer.}
    We have seen in \hardsectionref{17}{2} that, when the order $n$ of
    Bessel's differential equation is not an integer, the general
    solution of the equation is
    $$
    \alpha J_n(z) + \beta J_{-n}(z),
    $$
    where $\alpha$ and $\beta$ are arbitrary constants.

    % 
    % 370
    % 
    When, however, $n$ is an integer, we have seen that
    $$
    J_n(z) = (-)^n J_{-n}(z),
    $$
    and consequently the two solutions $J_n(z)$ and $J_{-n}(z)$ are
    not really distinct. We therefore require in this case to find
    another particular solution of the differential equation, distinct
    from $J_n(z)$ in order to have the general solution.

    We shall now consider the function
    $$
    \hardY_n(z)
    =
    2 \pi e^{n \pi i}
    \frac{J_n(z)\cos n\pi - J_{-n}(z)}{\sin 2n\pi},
    $$
    which is a solution of Bessel's equation when $2n$ is not an
    integer.
    The introduction of this function $\hardY_n(z)$ is due to
    Hankel\footnote{TODO}.

    When $n$ is an integer, $\hardY_n(z)$ is defined by the limiting
    form of this equation, namely
    \begin{align*}
      \hardY_n(z)
      =& \lim_{\eps\rightarrow 0}
      2\pi e^{(n+\eps)\pi i} \frac{ J_{n+\eps}(z) \cos(n\pi+\eps\pi) -
        J_{-n-\eps}(z)}{\sin 2(n+\eps)\pi}
      \\
      =& \lim_{\eps\rightarrow 0}
      \frac{2\pi e^{n\pi i}}{\sin 2\eps\pi}
      \thebrace{ J_{n+\eps}(z) \cdot(-)^n - J_{-n-\eps}(z)}
      \\
      =&
      \lim_{\eps\rightarrow 0}
      \eps^{-1}
      \thebrace{ J_{n+\eps}(z) - (-)^n J_{-n-\eps}(z)}.
    \end{align*}

    To express $\hardY_n(z)$ in terms of $W_{k,m}$ functions, we have
    recourse to the result of \hardsectionref{17}{5}, which gives
    \begin{align*}
      \hardY_n(z)
      = \lim_{\eps\rightarrow 0}
      \frac{\eps^{-1}}{ (2\pi z)^{\half}}
      \left[
        \thebrace{
          e^{\half(n+\eps+\half)\pi i} W_{0,n+\eps}(2iz)
          + e^{-\half(n+\eps+\half)\pi i} W_{0,n+\eps}(-2iz)
        }
      \right.
      \\
      \left.
        - (-)^n
        \thebrace{
          e^{\half(-n-\eps+\half)\pi i} W_{0,n+\eps}(2iz)
          + e^{-\half (-n-\eps+\half)\pi i} W_{0,n+\eps}(-2iz)
        }
      \right],
    \end{align*}
    remembering that $W_{k,m} = W_{k,-m}$.

    Hence, since\footnote{TODO}
    $\lim_{\eps\rightarrow 0} W_{0,n+\eps}(2iz) = W_{0,n}(2iz)$, we
    have
    $$
    \hardY_n(z)
    =
    \theparen{ \frac{\pi}{2z} }^{\half}
    \thebrace{
      e^{(\half n + \frac{3}{4}) \pi i} W_{0,n}(2iz)
      +
      e^{-(\half n + \frac{3}{4})\pi i} W_{0,n}(-2iz)
    }.
    $$

    This function ($n$ being an integer) is obviously a solution of
    Bessel's equation; it is called a \emph{Bessel function of the
      second kind}.

    Another function (also called a function of the second kind) was
    first used by TODO and by TODO; it is defined by the equation
    $$
    Y_n(z) = \frac{J_n(z) \cos n\pi - J_{-n}(z)}{\sin n\pi}
    = \frac{\hardY_n(z) \cos n\pi}{\pi e^{n\pi i}},
    $$
    %
    % 371
    %
    or by the limits of these expressions when $n$ is an integer. This
    function which exists for \emph{all} values of $n$ is taken as the
    canonical function of the second kind by TODO, and formulae
    involving it are generally but not always) simpler than the
    corresponding formulae involving Hankel's function.

    The asymptotic expansion for $Y_n(z)$, corresponding to that of
    \hardsectionref{17}{5} for $J_n(z)$, is that, when
    $\absval{\arg z} < \pi$ and $n$ is an integer,
    $$
    Y_n(z)
    \sim
    \theparen{ \frac{2}{\pi z} }^{\half}
    \thebracket{ \sin\theparen{z - \half n \pi - \frac{1}{4} \pi}
      \cdot U_n(z)
      + \cos\theparen{z - \half n \pi - \frac{1}{4}} \cdot V_n(z)
    },
    $$
    where $U_n(z)$ and $V_n(z)$ are the asymptotic expansions defined
    in \hardsectionref{17}{5}, their leading terms being $1$ and
    $(4n^2-1)/8z$ respectively.
    \begin{wandwexample}
      Prove that
      $$
      \hardY_n(z)
      =
      \frac{dJ_n(z)}{dn}
      -
      (-)^n \frac{dJ_{-n}(z)}{dn},
      $$
      where $n$ is made an integer after differentiation.
      TODO:attribution
    \end{wandwexample}
    \begin{wandwexample}
      Shew that if $\hardY_n(z)$ be defined by the equation of example
      TODO, it is a solution of Bessel's equation when $n$ is an integer.
    \end{wandwexample}
    \subsection{The ascending series for $\hardY_n(z)$.}
    The series of \hardsectionref{17}{6} is convenient for calculating
    $\hardY_n(z)$ when $\absval{z}$ is large. To obtain a convenient
    series for small values of $\absval{z}$, we observe that, since
    the ascending series for $J_{\pm (n+\eps)}(z)$ are uniformly
    convergent series of analytic functions\footnote{TODO} of $\eps$,
    each term may be expanded in powers of $\eps$ and this double
    series may then be arranged in powers of
    $\eps$ (TODO).

    Accordingly, to obtain $\hardY_n(z)$, we have to sum the
    coefficients of the first power of $\eps$ in the terms of the
    series
    $$
    \sum_{r=0}^{\infty}
    \frac{(-)^r (\half z)^{n+2r+\eps}}{r! \Gamma(n+\eps+r+1}
    -
    (-)^n
    \sum_{r=0}^{\infty}
    \frac{(-)^r (\half z)^{-n+2r -\eps}}{r! \Gamma(-n-\eps+r+1}
    $$

    Now, if $s$ be a positive integer or zero and $t$ a negative
    integer, the following expansions in powers of $\eps$ are valid:
    \begin{align*}
      \theparen{\half z}^{n+\eps+2r}
      =& \theparen{\half z}^{n+2r}
      \thebrace{1 + \eps\log\theparen{\half z} + \cdots},
      \\
      \frac{1}{\Gamma(s+\eps+1}
      =&
      \frac{1}{\Gamma(s+1)}
      \thebrace{
        1 - \eps \frac{\Gamma'(s+1)}{\Gamma(s+1)} + \cdots
      }
      \\
      =&
      \frac{1}{\Gamma(s+1)}
      \thebrace{
        1 - \eps\theparen{-\gamma + \sum_{m=1}^{s} m^{-1}} + \cdots
      },
      \\
      \frac{1}{\Gamma(t+\eps+1}
      =&
      - \frac{\sin(t+\eps)\pi}{\pi}
      \Gamma(-t-\eps)
      =
      (-)^{t+1} \eps \Gamma(-t) + \cdots,
    \end{align*}
    where $\gamma$ is Euler's constant (\hardsectionref{12}{1}).

    %
    % 372
    %
    Accordingly, picking out the coefficient of $\eps$, we see that
    \begin{align*}
    \hardY_n(z)
    =&
    \log\theparen{\half z}
    \thebracket{
      \sum_{r=0}^{\infty}
      \frac{(-)^r (\half z)^{n+2r}}{r! \Gamma(n+r+1)}
      + (-)^n \sum_{r=n}^{\infty} \frac{(-)^r (\half z)^{-n+2r}}{r! \Gamma(-n+r+1)}
    }
    \\
    &
    + \sum_{r=0}^{\infty}
    \frac{(-)^r (\half z)^{n+2r}}{r! \Gamma(n+r+1)}
    \theparen{\gamma - \sum_{m=1}^{n+r} m^{-1}}
    \\
    &
    + (-)^n
    \sum_{r=n}^{\infty}
    \frac{(-)^r (\half z)^{-n+2r}}{r! \Gamma(-n+r+1)}
    \theparen{\gamma - \sum_{m=1}^{r-n} m^{-1}}
    \\
    &
    + (-)^n
    \sum_{r=0}^{n-1}
    \frac{ (-)^r (\half z)^{-n+2r}}{r!}
    (-)^{r-n+1} \Gamma(n-r),
  \end{align*}
  and so
  \begin{align*}
    \hardY_n(z)
    =&
    \sum_{r=0}^{\infty}
    \frac{(-)^r (\half z)^{n+2r}}{r!(n+r)!}
    \thebrace{
      2 \log\theparen{\half z}
      + 2\gamma
      - \sum_{m=1}^{n+r} m^{-1}
      - \sum_{m=1}^r m^{-1}
    }
    \\
    & - \sum_{r=0}^{n-1} \frac{(\half z)^{-n+2r} (n-r-1)!}{r!}.
  \end{align*}
  When $n$ is an integer, fundamental solutions\footnote{TODO} of
  Bessel's equations, regular near $z=0$, are $J_n(z)$ and $Y_n(z)$ or
  $\hardY_n(z)$.

  Karl Neumann\footnote{TODO} took as the second solution the function
  $Y^{(n)}(z)$ defined by the equation
  $$
  Y^{(n)}(z)
  =
  \half \hardY_n(z) + J_n(z) \cdot (\log 2 - \gamma);
  $$
  but $Y_n(z)$ and $\hardY_n(z)$ are more useful for physical
  applications.
  \begin{wandwexample}
    Shew that the function $Y_n(z)$ satisfies the recurrence formula
    \begin{align*}
      n Y_n(z)
      =& \half z \thebrace{ Y_{n+1}(z) + Y_{n-1}(z)},
      \\
      Y_n'(z)
      =& \half \thebrace{ Y_{n-1}(z) - Y_{n+1}(z)  }.
    \end{align*}
    Shew also that Hankel's function $\hardY_n(z)$ and Neumann's
    function $Y^{(n)}(z)$ satisfy the same recurrence formulae.

    [These are the same as the recurrence formulae satisfied by $J_n(z)$.]
  \end{wandwexample}
  \begin{wandwexample}
    Shew that, when $\absval{\arg z} < \half \pi$,
    $$
    \pi Y_n(z)
    =
    \int_0^{\pi}
    \sin (z \sin\theta - n\theta) \dmeasure \theta
    -
    \int_0^{\infty}
    e^{-z \sinh\theta} \thebrace{e^{n\theta} + (-)^n e^{-n\theta}} \dmeasure\theta.
    TODO:addattribution
    $$
  \end{wandwexample}
  \begin{wandwexample}
    Shew that
    $$
    Y^{(0)}(z)
    =
    J_0(z) \log z
    + 2 \thebrace{ J_2(z) - \half J_4(z) + \frac{1}{3} J_6(z) - \cdots  }.
    $$
  \end{wandwexample}

\section{Bessel functions with purely imaginary argument.}
The function\footnote{TODO}
$$
I_n(z)
=
i^{-n}
J_n(iz)
=
\sum_{r=0}^{\infty}
  \frac{(\half z)^{n+2r}}{r!(n+r)!}
  $$
  %
  % 373
  %
  is of frequent occurrence in various branches of applied
  mathematics; in these applications $z$ is usually positive.

  The reader should have no difficulty in obtaining the following
  formulae:
  \begin{enumerate}
    \item $I_{n-1}(z) - I_{n+1}(z) = \frac{2n}{z} I_n(z)$.
    \item $\frac{d}{dz} \thebrace{z^n I_n(z)} = z^n I_{n-1}(z)$.
    \item $\frac{d}{dz} \thebrace{z^{-n} I_n(z)} = z^{-n} I_{n+1}(z)$.
    \item $\frac{d^2 I_n(z)}{dz^2} + \frac{1}{z} \frac{d I_n(z)}{dz} -
      \theparen{1 + \frac{n^2}{z^2}} I_n(z) = 0$.
    \item When $\realpart \theparen{n + \half} > 0$,
      $$
      I_n(z)
      =
      \frac{z^n}{2^n \Gamma(\half) \Gamma(n+\half)}
      \int_0^{\pi}
      \cosh (z\cos\phi) \sin^{2n} \phi \dmeasure \phi.
      $$
    \item When $-\frac{3}{2}\pi < \arg z < \half \pi$, the asymptotic
      expansion of $I_n(z)$ is
      \begin{align*}
        I_n(z)
        \sim
        \frac{ e^z}{ (2\pi z)^{\half}}
        \thebracket{1 + TODO}
        \\
        +
        \frac{ e^{-(n+\half) \pi i} e^{-z}}{ (2\pi z)^{\half} }
        \thebracket{TODO},
      \end{align*}
      the second series being negligible when $\absval{\arg z} < \half
      \pi$. The result is easily seen to be valid over the extended
      range $-\frac{3}{2} \pi < \arg z < \frac{3}{2} \pi$ is we write
      $e^{\pm (n+\half)\pi i}$ for $e^{-(n+\half)\pi i}$, the upper or
      lower sign being taken according as $\arg z$ is positive or negative.
    \end{enumerate}
  \end{document}